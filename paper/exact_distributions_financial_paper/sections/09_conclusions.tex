\section{Conclusion}\label{sec:conclusion}

We used correlated financial market data to extend a previous work
\cite{non_stationarity_fin_guhr} using three new exact multivariate
distributions proposed by Guhr and Schell \cite{exact_distributions_guhr}.
We applied a methodology, where a set of measured, non-stationary correlation
matrices are viewed as an ensemble for which is set up a random matrix model.
This ensemble is used to average the stationary multivariate amplitude
distributions measured on short time scales and thus obtained for large time
scales, multivariate amplitude distributions which feature heavy tails.

We used data from continuously traded companies in the S\&P 500 stock market
index, to find internal structures within the epochs. We chose all pairs of
returns, normalized within the epochs to mean $\mu = 0$ and $\sigma^{2} = 1$
and rotated and scaled the two-component returns. We showed that to a good
approximation, the returns can be multivariate Gaussian distributed or
multivariate algebraic distributed, depending on the epochs window length
choice.

After we reproduced the results for the Gaussian
\cite{non_stationarity_fin_guhr} case and added the result of the algebraic
case, we found an artifact in the original methodology. Thus, using simulated
time series we found the problem in the normalization within the epochs and
propose a solution normalizing the complete time series before the rotate and
scale step.

Having the internal structures within the epochs and after we solved the
artifact in the methodology, we compute four new distributions and compare them
with the rotated and scaled empirical data for the whole market. We found good
agreement with the tails of the distributions.