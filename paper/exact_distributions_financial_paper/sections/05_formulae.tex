\section{Exact multivariate amplitude distributions}
\label{sec:exact_distributions}

% In Sect. \ref{subsec:time_definition} we describe the physical time scale and
% the trade time scale. In Sect. \ref{subsec:trade_time} and Sect.
% \ref{subsec:physical_time} we define the trade and the physical time scales,
% respectively.

%%%%%%%%%%%%%%%%%%%%%%%%%%%%%%%%%%%%%%%%%%%%%%%%%%%%%%%%%%%%%%%%%%%%%%%%%%%%%%%
\subsection{Key concepts}\label{subsec:key_concepts}

Consider time series $S_{k} \left( t \right)$, $k = 1, 2, \ldots, K$ of stock
prices for $K$ companies. The values $S_{k} \left( t \right)$ are taken in
fixed time steps $\Delta t$. In general, the data contain an exponential
increase due to the drift. Thus, to measure the correlations independently of
this trend, it is better to use logarithmic differences instead of returns

\begin{equation}
    G_{k} \left( t \right) = \ln S_{k} \left( t + \Delta t \right) -
    \ln S_{k} \left(t \right) = \ln \frac{S_{k} \left( t + \Delta t \right)}
    {S_{k} \left(t \right)}.
\end{equation}
Anyway, logarithmic differences and returns almost coincide if the time steps
$\Delta t$ are sufficiently short \cite{subtle_nature,empirical_facts}
\begin{equation}
    G_{k} \left(t\right) \approx r_{k} \left(t\right)
    = \frac{S_{k} \left( t + \Delta t \right) - S_{k} \left( t \right)}
    {S_{k} \left( t \right)}.
\end{equation}
The returns are well known to have distributions with heavy tails, the smaller
$\Delta t$, the heavier
\cite{non_stationarity_fin_guhr}. Furthermore, the sample standard deviations
$\sigma_{k}$, referred to as volatilities, strongly fluctuate for different
time windows of the same length $T$
\cite{non_stationarity_fin_guhr,volatility_change}.

The mean of the logarithmic differences reads \cite{exact_distributions_guhr}

\begin{equation}
    \left\langle G_{k} \left( t \right) \right\rangle_{T} = \frac{1}{T}
    \sum_{t = 1}^{T} G_{k} \left( t \right).
\end{equation}
To compare the different $K$ companies, it is necessary to normalize the time
series. The normalized time series are defined by
\cite{exact_distributions_guhr,non_stationarity_fin_guhr}

\begin{equation}
    M_{k} \left( t \right) = \frac{G_{k} \left( t \right) - \left\langle
    G_{k} \left( t \right) \right\rangle} {\sqrt{\left\langle G_{k}^{2}
    \left( t \right) \right\rangle_{T} - \left\langle G_{k} \left( t \right)
    \right\rangle^2_{T}}},
\end{equation}
where
\begin{equation}
    \sigma_{k} = \sqrt{\left\langle G_{k}^{2}
    \left( t \right) \right\rangle_{T} - \left\langle G_{k} \left( t \right)
    \right\rangle^2_{T}}
\end{equation}
is the volatility of the $k$ company in the time window of length $T$. These
values can be viewed as the elements of a $K \times T$ rectangular matrix $M$.
With these normalizations and rescalings, it can be measured correlations in
such a way that all companies and all stocks are treated on equal footing.

The correlation coefficient for the stocks $k$ and $l$ is defined as
\cite{non_stationarity_fin_guhr}

\begin{equation}
    C_{kl} = \left\langle M_{k} \left( t \right) M_{l} \left( t \right)
    \right\rangle_{T} = \frac{1}{T} \sum_{t=1}^{T} M_{k} \left( t \right) M_{l}
    \left( t \right),
\end{equation}
which can be written as
\begin{equation}
    C_{kl} = \frac{\left\langle G_{k} \left( t \right) G_{l} \left( t \right)
    \right\rangle_{T} - \left\langle G_{k} \left( t \right) \right\rangle_{T}
    \left\langle G_{l} \left( t \right) \right\rangle_{T}}
    {\sigma_{k} \sigma_{l}}.
\end{equation}
The coefficients $C_{kl}$ are the elements of a $K \times K$ square matrix $C$,
the correlation matrix. The limiting values of these correlation coefficients
\begin{equation}
    C_{kl}^{\text{lim}} =
    \left\{
    \begin{array}{cc}
    +1 & \text{completely correlated}  \\
    0  & \text{completely uncorrelated}\\
    -1 & \text{completely anticorrelated}
    \end{array}
    \right. .
\end{equation}
The time average of $C_{kl}$ can be viewed as the matrix product of the
rectangular matrix $M$ ($K \times T$) with its transpose matrix $M^{\dagger}$
($T \times K$), divided by $T$. Thus, the correlation matrix can be written in
the form
\begin{equation}
    C = \frac{1}{T} M M^{\dagger}.
\end{equation}
The correlation matrix $C$ is real and symmetric. Using the correlation matrix
$C$ is possible to define the covariance matrix
\cite{credit_risk_guhr,portfolio_distributions_guhr,asset_correlations_guhr,stochastic_cov_guhr,exact_distributions_guhr}
\begin{equation}
    \Sigma = \sigma C \sigma ,
\end{equation}
where the diagonal matrix $\sigma$ contains the volatilities $\sigma_{k}$,
$k = 1, \ldots, K$.

%%%%%%%%%%%%%%%%%%%%%%%%%%%%%%%%%%%%%%%%%%%%%%%%%%%%%%%%%%%%%%%%%%%%%%%%%%%%%%%
\subsection{General considerations}\label{subsec:general_considerations}



%%%%%%%%%%%%%%%%%%%%%%%%%%%%%%%%%%%%%%%%%%%%%%%%%%%%%%%%%%%%%%%%%%%%%%%%%%%%%%%
\subsection{Gaussian-Gaussian distribution}\label{subsec:gaussian_gaussian}

%%%%%%%%%%%%%%%%%%%%%%%%%%%%%%%%%%%%%%%%%%%%%%%%%%%%%%%%%%%%%%%%%%%%%%%%%%%%%%%
\subsection{Gaussian-Algebraic distribution}\label{subsec:gaussian_algebraic}

%%%%%%%%%%%%%%%%%%%%%%%%%%%%%%%%%%%%%%%%%%%%%%%%%%%%%%%%%%%%%%%%%%%%%%%%%%%%%%%
\subsection{Algebraic-Gaussian distribution}\label{subsec:algebraic_gaussian}

%%%%%%%%%%%%%%%%%%%%%%%%%%%%%%%%%%%%%%%%%%%%%%%%%%%%%%%%%%%%%%%%%%%%%%%%%%%%%%%
\subsection{Algebraic-Algebraic distribution}\label{subsec:algebraic_algebraic}

