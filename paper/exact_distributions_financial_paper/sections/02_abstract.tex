\abstract{
      Correlated financial markets are perfect examples of highly
      non-stationary systems. In particular, sample averaged observables of
      time series as variances and correlation coefficients are continuously
      fluctuating, directly depending on the time window in which they are
      evaluated. Thus, models that describe the multivariate amplitude
      distributions of such systems are of considerable interest. Extending
      previous works, we apply a methodology, where a set of measured,
      non-stationary correlation matrices are viewed as an ensemble for which
      is set up a random matrix model. This ensemble is used to average the
      stationary multivariate amplitude distributions measured on short time
      scales and thus obtain for large time scales multivariate amplitude
      distributions which feature heavy tails. We explicitly use four cases,
      combining Gaussian and algebraic distributions to compare the
      distributions with empirical returns distributions using daily data from
      companies listed in the Standard \& Poor's (S\&P 500) stock market index
      in different periods of time. The comparison in the four cases with the
      empirical data reveals good agreement.
\PACS{
      {89.65.Gh}{Econophysics} \and
      {89.75.-k}{Complex systems} \and
      {05.10.Gg}{Statistical physics}
     } % end of PACS codes
} %end of abstract