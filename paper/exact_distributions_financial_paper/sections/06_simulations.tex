\section{Returns time series simulations}
\label{sec:simulations}

In Sect. \ref{subsec:gauss_alg_sim} we simulate returns time series following
a multivariate Gaussian and algebraic distribution. We test the influence of
the normalization within the epochs in the ergodicity defect in Sect.
\ref{subsec:norm_epochs_sim} and propose a solution for this defect in Sect.
\ref{subsec:norm_full_sim}. Finally we use empirical data to support our
simulation findings in Sect. \ref{subsec:emp_results}.

%%%%%%%%%%%%%%%%%%%%%%%%%%%%%%%%%%%%%%%%%%%%%%%%%%%%%%%%%%%%%%%%%%%%%%%%%%%%%%%
\subsection{Simulation of multivariate Gaussian and algebraic distributions}
\label{subsec:gauss_alg_sim}

In Sect. \ref{subsubsec:gauss_sim} we describe the methodology to simulate
multivariate Gaussian distributed returns time series and in Sect.
\ref{subsubsec:alg_sim} we describe the methodology to simulate multivariate
algebraic distributed returns time series.

%%%%%%%%%%%%%%%%%%%%%%%%%%%%%%%%%%%%%%%%%%%%%%%%%%%%%%%%%%%%%%%%%%%%%%%%%%%%%%%
\subsubsection{Multivariate Gaussian distributions}\label{subsubsec:gauss_sim}

To simulate the returns time series, we use a method \cite{drawing_dist} for
drawing a random vector $x$ from the $N$-dimensional multivariate Gaussian
distribution with mean vector $\mu$ and covariance matrix $C$. First, we create
a correlation matrix $C$ with $c = 1$ on its diagonal and $c = 0.3$ on its
non-diagonal entries. Then, we compute the eigenvalues and eigenvectors of the
correlation matrix, such that $C = U \Lambda U^{-1}$. We get a $z$ vector whose
components are drawn from a independent standard Gaussian distribution. Finally
we obtain the returns with the desired distribution as
\begin{equation}
    r = \mu + U \Lambda^{1/2} z
\end{equation}

In our case, the $r$ vector components are drawn from a normal distribution
with correlation matrix $C$ and $\mu$ vector zero.
With these method we want to obtain time series simulating the data matrix G
with dimensions $2 \times T$, where $T$ is the window length of the epochs.
These returns can be later normalized, rotated and aggregated to compare with
the behavior of the results in Sect. \ref{subsec:epochs}.  The goal of this
approach is that all simulations should show standard normal distributions.

\begin{figure}[htbp]
    \centering
    \includegraphics[width=0.6\columnwidth]
    {figures/06_epochs_sim_gauss_agg_ret_pairs_no_norm.png}
    \caption{Simulated aggregated rotated and scaled Gaussian distributed
             returns ($\tilde{r}$) for fixed covariance and $K=200$ without
             normalization, neither within the epochs nor for all the time
             series. $\Delta t = 1$ unit and epochs window  lengths
             $T=10, 25, 40, 55, 100$ units.}
    \label{fig:gauss_epochs_agg_ret_pairs_no_norm}
\end{figure}

In Fig. \ref{fig:gauss_epochs_agg_ret_pairs_no_norm} we simulate time series
for $K = 200$. Each time series is made of $200$ epochs to make them comparable
to the empirical data. We use epochs window lengths $T = 10, 25, 40, 55$. As
expected, as we draw the returns from a multivariate Gaussian distribution with
correlation matrix $C$ and $\mu$ vector zero, all the simulations show standard
Gaussian distribution behavior. Thus, this is our reference to check what is
introducing the ergodicity effect in the original method for the multivariate
Gaussian case.

%%%%%%%%%%%%%%%%%%%%%%%%%%%%%%%%%%%%%%%%%%%%%%%%%%%%%%%%%%%%%%%%%%%%%%%%%%%%%%%
\subsubsection{Multivariate algebraic distributions}\label{subsubsec:alg_sim}

To simulate returns time series drawn from multivariate algebraic
distributions, we use a similar approach as in Sect. \ref{subsubsec:gauss_sim}.
First, we create a correlation matrix $C$ with $c = 1$ on its diagonal and
$c = 0.3$ on its non-diagonal entries. From \cite{t_student_dist} we know that
\begin{equation}
    T = \left( S^{-1/2} \right)^{\dagger} X + M,
\end{equation}
where $T$ is a vector of length $K$. Then it is needed to repeat the following
steps to generate a data matrix where the columns are the $T$ vectors. $X$ is
drawn from a matrix variate normal distribution. Matrix $S$ is a Wishart
distributed covariance matrix without normalization and M is a parameter that
for this case is zero. With these in mind, we first generate $S$ as a
positive semi-definite matrix. To do this, we first create time series of
simulated data matrix $G$ with dimension $K \times \left(n + K - 1 \right)$
where $n$ is the a parameter of the degree of freedom and is connected to the
shape parameter $l$ as
\begin{equation}
    l = \frac{n + K}{2},
\end{equation}
where $K$ is the number of companies. These time series are generated by
calculating
\begin{equation}
    y = U \Lambda^{1/2} z
\end{equation}
with a fixed covariance matrix $\Sigma$. Vector $y$ is a column vector of $G$,
$z$ is a univariate standard normal distribution vector, $U$ has the
eigenvectors of $\Sigma$ as columns and the diagonal matrix $\Lambda$ contains
the eigenvalues of $\Sigma$. Thus, we compute
\begin{equation}
    S = G G^{\dagger}
\end{equation}
Then, we obtain $S^{1/2}$ as
\begin{equation}
    S^{1/2} = U_{S} \Lambda_{S}^{1/2} U_{S}^{\dagger},
\end{equation}
since
\begin{align}
    S &= S^{1/2} S^{1/2} \\
    &= U_{S} \Lambda_{S}^{1/2} U_{S}^{\dagger}
    U_{S} \Lambda_{S}^{1/2} U_{S}^{\dagger}\\
    &= U_{S} \Lambda_{S} U_{S}^{\dagger}
\end{align}
We generate $X$ as
\begin{equation}
    X = \sqrt{m} z
\end{equation}
where $z$ is a univariate standard normal distribution vector of length $K$ and
$m$ is the variance.

These returns can be later normalized, rotated and aggregated to compare with
the behavior of the results in Sect. \ref{subsec:epochs}.  The goal of this
approach is that all simulations should show standard algebraic distributions.

\begin{figure}[htbp]
    \centering
    \includegraphics[width=0.6\columnwidth]
    {figures/06_epochs_sim_alg_agg_ret_pairs_no_norm.png}
    \caption{Simulated aggregated rotated and scaled algebraic distributed
             returns ($\tilde{r}$) for fixed covariance and $K=200$ without
             normalization, neither within the epochs nor for all the time
             series. $\Delta t = 1$ unit and epochs window  lengths
             $T=10, 25, 40, 55, 100$ units.}
    \label{fig:alg_epochs_agg_ret_pairs_no_norm}
\end{figure}

In Fig. \ref{fig:alg_epochs_agg_ret_pairs_no_norm} we simulate time series for
$K = 200$. Each time series is made of $200$ epochs to make them comparable
to the empirical data. We use epochs window lengths $T = 10, 25, 40, 55$. We
can see an interesting behavior in the algebraic case, where for small epochs
window lengths, the tails are similar to the Gaussian distribution, and as the
epochs window lengths grow, the simulations reveals good agreement with the
algebraic distribution. Thus, this is our reference to check what is
introducing the ergodicity effect in the original method for the multivariate
algebraic case.

%%%%%%%%%%%%%%%%%%%%%%%%%%%%%%%%%%%%%%%%%%%%%%%%%%%%%%%%%%%%%%%%%%%%%%%%%%%%%%%
\subsection{Normalization within the epochs}
\label{subsec:norm_epochs_sim}

Now, to check the normalization within the epochs, we simulate the returns,
normalize each epoch to mean $\mu = 0$ and variance $\sigma^{2} = 1$ and
finally repeat the procedure of rotate, scale and aggregate.

For the Gaussian case, with the simulated pair returns time series, we proceed
to normalize the epoch, we evaluate the $2 \times 2$ sample covariance matrix
and diagonalize it. We rotate the two-component returns vectors into the
eigenbasis of the correlation matrix and normalize the axis with the
eigenvalues. Finally we aggregate all the components into a single univariate
distribution.

\begin{figure}[htbp]
    \centering
    \includegraphics[width=0.6\columnwidth]
    {figures/06_epochs_sim_gauss_agg_ret_pairs_norm.png}
    \caption{Simulated aggregated rotated and scaled Gaussian distributed
             returns ($\tilde{r}$) for fixed covariance and $K=200$ with
             normalization within the epochs. $\Delta t = 1$ unit and epochs
             window lengths $T=10, 25, 40, 55, 100$ units.}
    \label{fig:epochs_gauss_agg_ret_pairs_norm}
\end{figure}

As we can see in Fig. \ref{fig:epochs_gauss_agg_ret_pairs_norm}, the ergodicity
defect clearly appears for an epoch window length $T = 10$. As the epoch window
length grows, the ergodicity defect starts to disappear. We could even argue,
that with an epoch window length greater or equal to $T = 25$, we already are
close enough to the Gaussian distribution, which were the objective of these
simulations. Furthermore, with large epoch windows lengths we can confirm that
the ergodicity defect disappear, as it can be seen with $T = 100$

Something similar happens with the algebraic case. We again simulate the
returns, normalize each epoch to mean $\mu = 0$ and variance $\sigma^{2} = 1$
and finally repeat the procedure of rotate, scale and aggregate. In Fig.
\ref{fig:epochs_alg_agg_ret_pairs_norm} can be seen how the ergodicity effect
appears again. It seems to have a large effect in epochs window lengths with
small values. Thus, we need to find an alternative to solve this problem.

\begin{figure}[htbp]
    \centering
    \includegraphics[width=0.6\columnwidth]
    {figures/06_epochs_sim_alg_agg_ret_pairs_norm.png}
    \caption{Simulated aggregated rotated and scaled Gaussian distributed
             returns ($\tilde{r}$) for fixed covariance and $K=200$ with
             normalization within the epochs. $\Delta t = 1$ unit and epochs
             window lengths $T=10, 25, 40, 55, 100$ units.}
    \label{fig:epochs_alg_agg_ret_pairs_norm}
\end{figure}

%%%%%%%%%%%%%%%%%%%%%%%%%%%%%%%%%%%%%%%%%%%%%%%%%%%%%%%%%%%%%%%%%%%%%%%%%%%%%%%
\subsection{Normalization complete return time series}
\label{subsec:norm_full_sim}

We already showed that the ergodicity effect is directly related with the
normalization of the time series. To try to solve this issue, instead of
normalize the time series within each epoch, we normalize the complete time
series and then proceed to rotate, scale and aggregate.

\begin{figure}[htbp]
    \centering
    \includegraphics[width=0.6\columnwidth]
    {figures/06_epochs_sim_gauss_ts_norm.png}
    \caption{Simulated aggregated rotated and scaled Gaussian distributed
             returns ($\tilde{r}$) for fixed covariance and $K=200$ with
             normalization for the complete time series. $\Delta t = 1$ unit
             and epochs window lengths $T=10, 25, 40, 55, 100$ units.}
    \label{fig:epochs_gauss_agg_ret_pairs_norm_full_ts}
\end{figure}

\begin{figure}[htbp]
    \centering
    \includegraphics[width=0.6\columnwidth]
    {figures/06_epochs_sim_alg_ts_norm.png}
    \caption{Simulated aggregated rotated and scaled Gaussian distributed
             returns ($\tilde{r}$) for fixed covariance and $K=200$ with
             normalization for the complete time series. $\Delta t = 1$ unit
             and epochs window lengths $T=10, 25, 40, 55, 100$ units.}
    \label{fig:epochs_alg_agg_ret_pairs_norm_full_ts}
\end{figure}

%%%%%%%%%%%%%%%%%%%%%%%%%%%%%%%%%%%%%%%%%%%%%%%%%%%%%%%%%%%%%%%%%%%%%%%%%%%%%%%
\subsection{Empirical results}
\label{subsec:emp_results}