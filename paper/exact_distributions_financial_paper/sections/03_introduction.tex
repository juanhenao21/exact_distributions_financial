\section{Introduction}\label{sec:introduction}

A bunch of different characteristics in complex systems can be tracked down to
non-stationarity \cite{non_stat_1,non_stat_2}. These systems lack any kind of
equilibrium \cite{comp_sys_1,comp_sys_2,comp_sys_3,comp_sys_4}. Financial
markets are perfect examples of non-stationarity as they fluctuate considerably
in time. In general, the business relations between companies and agents can
change due to market expectations. During state of crisis, non-stationarity
becomes dramatic
\cite{state_crisis_1,state_crisis_2,state_crisis_3,state_crisis_4,state_crisis_5,state_crisis_6,state_crisis_7}.

The fluctuation of the correlations induces generic features in financial time
series, where we showed that these fluctuations lift the tails of the
multivariate amplitude distributions, making them heavy-tailed
\cite{non_stationarity_fin_guhr,exact_distributions_guhr}.

Our goal is to use the analytical results for the multivariate distributions of
amplitudes, measured as time series in correlated, non-stationary financial
markets and provide quantitative measures for the degree of non-stationarity in
the correlations using the methodology first proposed in
\cite{non_stationarity_fin_guhr} and extended in
\cite{exact_distributions_guhr}. These amplitudes refer to the stock price
changes for the entire market.

We carry out a detailed data analysis that exposes generic features. Then we
use a random matrix model to explain them. We show that non-stationarity of the
correlations leads to heavy tails in the multivariate return distribution and
finally, we use the approach in
\cite{non_stationarity_fin_guhr,exact_distributions_guhr} to map
a non-invariant situation to an effectively invariant one.

A remarkable feature of the multivariate distributions we use to compare with
the financial data, is that eventually, they are of closed form or involve only
single integrals. Moreover, they use a low number of free parameters: one
measuring how strongly the non-stationary correlations fluctuate, and one or
two shape parameters for the tails. All the other parameters can be directly
measured from the data \cite{exact_distributions_guhr}.

Random matrix models \cite{random_matrix_1,random_matrix_2} fall into two
classes: (I) The ensemble is fictitious. It comes into play via an ergodicity
argument only. (II) The ensemble really exist and can be identified in the
system. The issue of ergodicity does not arise. It is conceptually important
that we here use a random matrix model in class (II) which may be seen as a new
interpretation of the Wishart model and generalizations thereof for random
covariance or correlation matrices \cite{wishart}. In finance there are
numerous random matrix applications
\cite{matrix_fin_01,matrix_fin_02,matrix_fin_03,matrix_fin_04,matrix_fin_05,matrix_fin_06,matrix_fin_07,matrix_fin_08,matrix_fin_09,matrix_fin_10,matrix_fin_11,matrix_fin_12,matrix_fin_13}
including non-Gaussian ensembles. To the best of our knowledge, all of them
fall into class (I) and focus on other observables. The distributions we use
arrive at rather universal and generic results, supporting view that
non-stationarities can lead to universal features.

The paper is organized as follows: in Sect. \ref{sec:data_set} we present our
data set of stocks and define the analyzed time intervals. We define the key
concepts and the general considerations in Sect. \ref{sec:exact_distributions}.
In Sect. \ref{sec:comparison_returns} we show compare the theoretical
distributions with the aggregated distribution of returns. Our conclusions
follow in Sect. \ref{sec:conclusion}.